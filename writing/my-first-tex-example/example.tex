\documentclass[11pt,a4paper]{article}
\usepackage[utf8]{inputenc}
\usepackage[T1]{fontenc}
\usepackage[english]{babel}
\usepackage{amsmath}
\usepackage{amsfonts}
\usepackage{amssymb}
\usepackage{graphicx}
\usepackage{booktabs}
\usepackage{multirow}
\usepackage{hyperref}
\usepackage{natbib}
\usepackage{setspace}
\usepackage{geometry}

% Page layout
\geometry{margin=1in}
\setstretch{1.15}

% Title information
\title{Title of Your Research Article}
\author{
    Author One\thanks{Corresponding author: author.one@email.com} \\
    Institution One \\
    \and
    Author Two \\
    Institution Two
}
\date{\today}

\begin{document}

% Title section
\maketitle

% Abstract
\begin{abstract}
This is a brief summary of your research article. It should concisely state the purpose, methods, key findings, and conclusions. Typically 150-250 words. The abstract should be self-contained and understandable without reference to the main text.
\end{abstract}

% \keywords{keyword one, keyword two, keyword three}

% Main content
\section{Introduction}
\label{sec:introduction}

Start your introduction here. Provide context and background for your research. State the research problem clearly. 

%Cite previous work using \citet{smith2020} for textual citations or \citep{jones2021} for parenthetical citations.

The introduction should end with a clear statement of your research objectives or hypotheses.

\section{Literature Review}
\label{sec:literature}

Review relevant literature in your field. Discuss key theories, previous findings, and identify research gaps.

\section{Methodology}
\label{sec:methodology}

\subsection{Data Collection}
Describe your data sources and collection methods.

\subsection{Analytical Methods}
Explain your analytical approach. You can include equations:

\begin{equation}
\label{eq:example}
y = \beta_0 + \beta_1 x_1 + \beta_2 x_2 + \epsilon
\end{equation}

\section{Results}
\label{sec:results}

Present your findings. Use tables and figures to illustrate key results.

\begin{table}[htbp]
\centering
\caption{Descriptive Statistics}
\label{tab:descriptives}
\begin{tabular}{lcccc}
\toprule
Variable & Mean & SD & Min & Max \\
\midrule
Variable 1 & 10.5 & 2.3 & 5.0 & 15.0 \\
Variable 2 & 25.1 & 5.7 & 12.0 & 40.0 \\
\bottomrule
\end{tabular}
\end{table}

\begin{figure}[htbp]
\centering
\includegraphics[width=0.8\textwidth]{example-image}
\caption{Example figure showing important results}
\label{fig:results}
\end{figure}

\section{Discussion}
\label{sec:discussion}

Interpret your results in the context of existing literature. Discuss implications, limitations, and unexpected findings.

\section{Conclusion}
\label{sec:conclusion}

Summarize your main findings and their significance. Suggest directions for future research.

% References
%\bibliographystyle{apa}
%\bibliography{references}

% Appendices (optional)
\appendix
\section{Supplementary Materials}
\label{app:supplementary}

Additional materials like questionnaires, code snippets, or extended results.

\end{document}